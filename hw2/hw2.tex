\documentclass[twoside, letter]{article}
\setlength{\oddsidemargin}{0.01 in}
\setlength{\evensidemargin}{0.01 in}
\setlength{\topmargin}{-0.6 in}
\setlength{\textwidth}{6.5 in}
\setlength{\textheight}{8.5 in}
\setlength{\headsep}{0.75 in}
\setlength{\parindent}{0 in}
\setlength{\parskip}{0.1 in}

\usepackage{natbib}
\usepackage{hyperref}
\usepackage{listings}
\usepackage{xcolor}
\definecolor{codegreen}{rgb}{0,0.6,0}
\definecolor{codegray}{rgb}{0.5,0.5,0.5}
\definecolor{codepurple}{rgb}{0.58,0,0.82}
\definecolor{backcolour}{rgb}{0.95,0.95,0.92}
\usepackage{pdfpages}

\lstdefinestyle{mystyle}{
    backgroundcolor=\color{backcolour},   
    commentstyle=\color{codegreen},
    keywordstyle=\color{magenta},
    numberstyle=\tiny\color{codegray},
    stringstyle=\color{codepurple},
    basicstyle=\ttfamily\footnotesize,
    breakatwhitespace=false,         
    breaklines=true,                 
    captionpos=b,                    
    keepspaces=true,                 
    numbers=left,                    
    numbersep=5pt,                  
    showspaces=false,                
    showstringspaces=false,
    showtabs=false,                  
    tabsize=2
}

\usepackage{caption}
\lstset{style=mystyle}
%
% ADD PACKAGES here:
%



\usepackage{amsmath,amsfonts,amssymb,graphicx,mathtools,flexisym}

%
% The following commands set up the lecnum (lecture number)
% counter and make various numbering schemes work relative
% to the lecture number.
%
\newcounter{lecnum}
\renewcommand{\thepage}{\thelecnum-\arabic{page}}
\renewcommand{\thesection}{\thelecnum.\arabic{section}}
\renewcommand{\theequation}{\thelecnum.\arabic{equation}}
\renewcommand{\thefigure}{\thelecnum.\arabic{figure}}
\renewcommand{\thetable}{\thelecnum.\arabic{table}}

%
% The following macro is used to generate the header.
%
\newcommand{\lecture}[4]{
   \pagestyle{myheadings}
   \thispagestyle{plain}
   \newpage
   \setcounter{lecnum}{#1}
   \setcounter{page}{1}
   \noindent
   \begin{center}
   \framebox{
      \vbox{\vspace{2mm}
    \hbox to 6.28in { {\bf STA 141C - Big Data \& High Performance Statistical Computing
	\hfill Spring 2023} }
       \vspace{4mm}
       \hbox to 6.28in { {\Large \hfill Homework \# 2 \hfill} }
       \vspace{2mm}
       %\hbox to 6.28in { {\it Lecturer: #3 \hfill Scribes: #4} }
       \hbox to 6.28in { {\it Lecturer: #2 \hfill  Due May 12, 2022} }
      \vspace{2mm}}
   }
   \end{center}
   \markboth{Homework #1: #2}{Homework #1: #2}

  % {\bf Note}: {\it LaTeX template courtesy of UC Berkeley EECS dept.}

}
%
% Convention for citations is authors' initials followed by the year.
% For example, to cite a paper by Leighton and Maggs you would type
% \cite{LM89}, and to cite a paper by Strassen you would type \cite{S69}.
% (To avoid bibliography problems, for now we redefine the \cite command.)
% Also commands that create a suitable format for the reference list.
\renewcommand{\cite}[1]{[#1]}
\def\beginrefs{\begin{list}%
        {[\arabic{equation}]}{\usecounter{equation}
         \setlength{\leftmargin}{2.0truecm}\setlength{\labelsep}{0.4truecm}%
         \setlength{\labelwidth}{1.6truecm}}}
\def\endrefs{\end{list}}
\def\bibentry#1{\item[\hbox{[#1]}]}

%Use this command for a figure; it puts a figure in wherever you want it.
%usage: \fig{NUMBER}{SPACE-IN-INCHES}{CAPTION}
\newcommand{\fig}[3]{
			\vspace{#2}
			\begin{center}
			Figure \thelecnum.#1:~#3
			\end{center}
	}
% Use these for theorems, lemmas, proofs, etc.
\newtheorem{theorem}{Theorem}[lecnum]
\newtheorem{lemma}[theorem]{Lemma}
\newtheorem{proposition}[theorem]{Proposition}
\newtheorem{claim}[theorem]{Claim}
\newtheorem{corollary}[theorem]{Corollary}
\newtheorem{definition}[theorem]{Definition}
\newenvironment{proof}{{\bf Proof:}}{\hfill\rule{2mm}{2mm}}

% **** IF YOU WANT TO DEFINE ADDITIONAL MACROS FOR YOURSELF, PUT THEM HERE:

\newcommand\E{\mathbb{E}}
\newcommand{\bitm}{\begin{itemize}}
\newcommand{\eitm}{\end{itemize}}
\newcommand{\blst}{\begin{lstlisting}}
\newcommand{\elst}{\end{lstlisting}}
\newcommand{\bfig}{\begin{figure}}
\newcommand{\efig}{\end{figure}}


\renewcommand{\thesection}{\arabic{section}}
\renewcommand{\thesubsection}{\thesection.\arabic{subsection}}
\renewcommand{\thesubsubsection}{\thesubsection.\arabic{subsubsection}}

\begin{document}
%FILL IN THE RIGHT INFO.
%\lecture{**LECTURE-NUMBER**}{**DATE**}{**LECTURER**}{**SCRIBE**}
\lecture{2}{Bo Y.-C. Ning}{Bo Y.-C. Ning}{}
%\footnotetext{These notes are partially based on those of Nigel Mansell.}

% **** YOUR NOTES GO HERE:

Due {\bf May 12, 2023} by 11:59pm. 

This homework has two major goals: 1) compare the computational speeds for solving linear equation using QR, Cholesky, and GE/LU methods; 2) learn how to implement parallel computing in R or python. 

Directions:
\begin{enumerate}
\item Submit your homework using the file name "{\bf LastName\_FirstName\_hw2}"

\item Answer all questions with complete sentences. For proofs, please provide the intermediate steps.

\item Your code should be readable; writing a piece of code should be compared to writing a page of a book. Adopt the {\bf one-statement-per-line} rule. Consider splitting a lengthy statement into multiple lines to improve readability. (You will lose one point for each line that does not follow the one-statement-per-line rule)

\item To help understand and maintain code, you should always add comments to explain your code. (homework with no comments will receive 0 points). For a very long comment, break it into multiple lines.

\item Submit your final work with one {\bf .pdf} (or {\bf .html}) file to Canvas. I encourage you to use \href{http://www.docs.is.ed.ac.uk/skills/documents/3722/3722-2014.pdf}{\LaTeX} for writing equations and proofs. Handwriting is acceptable, you have to scan it and then combine it with the coding part into a single .pdf (or .html) file. Handwriting should be clean and readable.

\item For $\mathsf{Jupyter \ Notebook}$ users, put your answers in new cells after each exercise. You can make as many new cells as you like. Use code cells for code and Markdown cells for text. 

\item This assignment will be graded for correctness. 
\end{enumerate}





%%%%%%%%%%%%%%%%%%%%%%%%%%%%%%%%%%%%%%%%%%%%%%%
%%%%%%%%%%%%%%%%%%%%%%%%%%%%%%%%%%%%%%%%%%%%%%%
%%%%%%%%%%%%%%%%%%%%%%%%%%%%%%%%%%%%%%%%%%%%%%%
{\Large \bf Questions:}
\begin{enumerate}
\item Read in the `longley.dat' with the response (number of people employed) in the first
column and six explanatory variables in the other columns (GNP implicit price deflator,
Gross National Product, number of unemployed, number of people in the armed forces,
``noninstitutionalized'' population \% 14 years of age, year). Include an intercept in you
model.

\item Assuming linear model $y \sim N(X\beta, \sigma^2 I)$, compute 
1) regression coefficients $\hat \beta = (X'X)^{-1} X'y$,
2) standard errors of $\hat \beta$, which is $\hat \sigma \sqrt{\text{diag}({(X'X)^{-1}})}$, 
and 3) variance estimate $\hat \sigma^2 = (y - X\hat \beta)'(y - X\hat \beta)/(n - p)$ using following methods: GE/LU decomposition, Cholesky decomposition, and QR decomposition, and compare the computation speed for each method. Please compute them directly using numerical linear algebra functions; you can use the ``black-box'' function (e.g., $\mathsf{lm()}$ in R or $\mathsf{sklearn.linear\_model.LinearRegression}$ in python) {\bf only to check your
results.} (Hint: $\mathsf{chol2inv()}$ function in R computes the inverse of a matrix from its Cholesky
factor. In python, you may try $\mathsf{cho\_solve()}$)


\item One popular regularization method is the ridge regression, which estimates regression
coefficients by minimizing a penalized least squares criterion
$$
\frac{1}{2} \|y - X\beta\|_2^2 + \frac{\lambda}{2} \|\beta\|_2^2,
$$
show that the ridge solution is given by 
$$
\hat \beta_\lambda = (X'X + \lambda I_p)^{-1} X'y.
$$

\item Compute the ridge regression estimates $\hat \beta_\lambda$ at a set of
different values of $\lambda$ (e.g., 0, 1, 2, $\dots$, 100) by solving it as a least squares problem. 
Plot the $\ell_2$-norm of the ridge coefficients $\|\hat \beta_\lambda\|$ as a function of $\lambda$.
You can use either QR or Cholesky method.

\item Implement your code using parallel computing.

\item Find out which method is the $\mathsf{lm()}$ function in R is
using? And which algorithm is being used?
{\bf Or} find out which method is the linear regression function (there are multiple, but only need to choose one) in numpy/scipy is using? And which algorithm is being used? 
\end{enumerate}


%%%%%%%%%%%%%%%%%%%%%%%%%%%%%%%%%%%%%%%%
%%%%%%%%%%%%%%%% Bibliography %%%%%%%%%%%%%%%%%


 \end{document}